%% LyX 2.1.3 created this file.  For more info, see http://www.lyx.org/.
%% Do not edit unless you really know what you are doing.
\documentclass[english]{article}
\usepackage[T1]{fontenc}
\usepackage[utf8]{luainputenc}
\usepackage{amsmath}
\usepackage{amssymb}
\usepackage{graphicx}
\usepackage{subscript}
\usepackage{babel}
\begin{document}

\title{Space, Time, and Sociability: Predicting Future Interactions}


\author{Christoph Stich}

\maketitle

\section{Introduction (1000 words)}

Researchers have long studied the relationship between between space,
time, and social structure as space, time and the social realm are
intrinsically linked. For the most part people do not aimlessly amble,
nor is happenstance usually the reason people spend time at places.
People commute to work every day, they meet their friends at a bar
after work, or they go on a date with their partner.

With the increased dissemination of GPS, mobile technologies, and
social networks new avenues for research have opened up in recent
years. {[}write some more{]}.

A number of important question however remain unstudied. For example,
can we use chronological and spatial information to predict future
interactions. So far work that has taken geographic features into
account has focused on social structure and not interactions (quotes),
whereas predicting interactions (Yang 2013) has not incorporated geographic
information. 

While the importance of time and space for social processes has been
acknowledged by various authors () it is unclear whether place in
itself has genuine predictive power or is simply confounded with the
geographic embeddedness of social interactions. In other words does
place matter or is it actually the people you happen to meet at different
places. 

Another open question is whether the type of relationship between
nodes has any influence on the predictability of interactions. One
could assume that meetings between colleagues are highly predictable
as they both share a specific physical location---work---that both
visit with high regularity. On the other hand one could assume that
meeting your social ties is driven by a much more complex process
and thus appear to be less regular.

To address those questions, we propose to phrase the question of whether
two people will meet in a given time period as a link prediction problem
in a social interaction graph and assess the predictive power of time,
place, and social variable have for different scenarios. The rest
of the paper is organized as follows: We discuss related work in section
\ref{sec:Related-Work} and formally define our problem in section
\ref{sec:Problem-Definition}. Section \ref{sec:Social-Interaction-Graph}
describes the two datasets we use in our paper as well as explores
the social interaction graph. Section \ref{sec:Prediction} details
the setup for our prediction task, whereas section \ref{sec:Findings}
discusses our findings. Last but not least we discuss the implications
of our findings in section \ref{sec:Conclusion}.


\section{\label{sec:Related-Work}Related Work}

Predicting links in a graph is a well studied problem in computation
(see the review paper).

{[}Write paragraph(s) about space and social{]}.

{[}Write a paragraph(s) about time and social. Regular patterns of
visits{]}.


\section{\label{sec:Problem-Definition}Problem Definition}


\subsection{Social interaction graph}

We phrase the problem of predicting interaction as a link prediction
problem in a time-varying, labeled network G that represents interactions.
We define interaction any physical proximity measured by a strong
bluetooth measurement. We use of -80 as Vedran () has shown this to
be a reliable cut-off value for close and unobstructed physical proximity
{[}expand a bit{]}. For each timepoint t we build a graph $G_{t}=(V_{t},\,E_{t},)$,
where \emph{V\textsubscript{t}} are the set of students at time \emph{t}
and \emph{E\textsubscript{t }}the set of links between them. Each
edge $e\in E$ has also an associated weight \emph{$w(e)\text{\ensuremath{\in}}\mathbb{Z}$}
representing the amount of interaction between any two nodes $u,v\in V$
in the previous period $\triangle T$. Let $W_{t}$ be the distribution
of assigned weights at a given timepoint t and \emph{W }be the distribution
of all weights over all possible timepoints. We also assign a label\emph{
$L(w(e))\text{\ensuremath{\in}}\left\{ 0,1,2,3\right\} $} to each
$e\in E_{t}$ to be able to use state-of-art algorithms in link-prediction
that use class labels as domain. The labels $L(e)$ are based on the
amount of interaction in the preceding time period and the cut-off
values for each label are defined in relation to the quartils of \emph{W}.
Let\emph{ L(e) }now be $\begin{cases}
0: & m=0\\
1: & 0<m<Q_{1}(W)\\
2: & Q_{1}(W)<m<Q_{3}(W)\\
3: & Q_{3}(W)<m
\end{cases}$ and \emph{m} the observed time of interaction between any two $u,v\in V$
during $\triangle T$. These particular labels were chosen to be able
to distinguish between \emph{weak ties} ($l=1$), \emph{ties} ($l=2$),
and \emph{strong ties} ($l=3$).


\subsection{Link-prediction}

In a human interaction network $G_{t}=(V_{t},\,E_{t})$, the multi-class
link prediction task is to predict $L(e)$ at time $t+\triangle t$,
where $e(u,\,v)\not\in E_{t}$ and $u,\,v\,\text{\ensuremath{\in}}\,V_{t}$. 

Intuitively, we are trying to predict who will meet whom and for who
long during period $\triangle t$. Formulating the problem this way
has the advantage of including link dissolution---a not well studied
problem in link-prediction (quote survey)---quite naturally in the
problem definition. This is equivalent to predicting the labeled,
network structure of $G_{t+\triangle t}$.


\section{\label{sec:Social-Interaction-Graph}The week, geographic context,
and social interactions}


\subsection{Datasets}

Our data consists of two datasets. {[}Write something about the datasets{]}{[}Descriptive
plots of the behavioral dynamics{]}

{[}Data exploration{]}


\subsection{Descriptive parts}

Because the literature reliably suggests a link between geography,
time, and the social realm, we investigate our data to check whether
we can observe a relationship between those factors. If there is a
relationship between space, time, and social interactions we should
be able to observe differences in how people visit those places and
in how they socialize at different geographic settings. 


\subsubsection{Geographic contexts}

Based on Oldenburg's seminal paper () we develop a definition of different
geographic contexts for our study, whose influence on link-prediction
we are interested in. {[}write a short paragraph about oldenburg{]}
Analogous to Oldenburg we distinguish between several different geographic
settings a student can be in: the \emph{home, }the\emph{ university,
}a \emph{third place}, and \emph{other.} We infer the home location
for each student by clustering all his or her location measurements
between 11PM and 4AM using DBSCAN () into the set of clusters \emph{C.}
We then select $max(|c|),c\in C$ as a student's home location. For
assigning students to the \emph{university }context we mapped the
campus of their university and checked whether students where within
50 meters of the campus. To infer the \emph{third places }for each
student we first construct the set of all the stop locations a student
would visit $L_{stop}$. For each $l\in L_{stop}$ we can also observe
the amount of time \emph{t(l)} a student spends there. We then define
a \emph{third place }as any $l\in L_{stop}$that fulfills the following
inequality: $t(l)/\sum t(l_{i})\geq0.1$ and is not either \emph{home
}or \emph{university}. A \emph{third place }is thus any location where
a student spends at least 10\% of his time that is neither \emph{home
}nor \emph{university. }Lastly any other $l\in L_{stop}$ is classified
as \emph{other.} 


\subsubsection{Activity pattern and social interactions}

Looking at the aggregated weekly activity pattern of all students
one can {[}say something about the figure{]}. First, we observe that
during weekdays the two clearly dominating settings are \emph{home
}or \emph{university }and the pattern is remarkably similar for all
weekdays. Second, during weekends the dominant setting is \emph{third
Places }while almost no one visits the \emph{university.} Also people
explore a lot of \emph{other }locations during the weekend as well.
Thus, the behavior of students is consequentially different on weekdays
than from weekends. Thus, it seems to safe to conclude that there
is indeed a relationship between geographic setting and time as the
observed behavior is too regular to be product of chance.

\begin{figure}
\protect\caption{Weekly aggregated activity pattern\label{fig:Activity-pattern}}
\includegraphics[width=1\textwidth,bb = 0 0 200 100, draft, type=eps]{/home/christoph/raman/home/christoph/linkPrediction/results/actitivityPattern.png}
\end{figure}


\begin{figure}
\protect\caption{$\overline{x}_{peers}$}


\includegraphics[width=1\textwidth,bb = 0 0 200 100, draft, type=eps]{/home/christoph/raman/home/christoph/linkPrediction/results/meetingsDistribution.png}
\end{figure}
When looking at when people meet as in figure \ref{fig:average-peers}
one can clearly distinguish between weekdays and weekends and between
days and nights. However, when looking at the different geographic
contexts the picture is less clear. Figure \ref{fig:deviation-average-peers}
shows the absolute deviation from the average amount of peers present
for each geographic setting. 

\begin{figure}
\protect\caption{\label{fig:deviation-average-peers}Deviation from $\overline{x}_{peers}$}
\includegraphics[width=1\textwidth,bb = 0 0 200 100, draft, type=eps]{/home/christoph/raman/home/christoph/linkPrediction/results/meetingsDistributionAllContexts.png}
\end{figure}


While one can spot differences their interpretation is less straight
forward and one cannot determine whether the differences are due to
chance. Lag plots are often used to determine whether a series is
random or not (find a quote). An analysis of the lags in figure X
reveals that the variations are not random as there are clearly linear
patterns visible for all geographic contexts

\begin{figure}
\protect\caption{\label{fig:Lag-plots-deviations}Lag plots deviations of $\overline{x}_{peers}$}


\includegraphics[width=1\textwidth,bb = 0 0 200 100, draft, type=eps]{/home/christoph/raman/home/christoph/linkPrediction/results/lag_peers.png}
\end{figure}


\begin{figure}
\protect\caption{$\overline{x}_{edge\,Life}$}
\includegraphics[width=1\textwidth,bb = 0 0 200 100, draft, type=eps]{/home/christoph/raman/home/christoph/linkPrediction/results/edge_life_contexts.png}
\end{figure}


If one looks at the the average edge life of each interaction between
students for the different settings one can see that students interact
for longer at home, at a third place, and at an other place then at
university. At the university students interact for roughly 90 minutes,
whereas the interactions at the other settings are significantly longer.
Also the length of interactions is dependent on the time of the day.
To offer a blunt interpretation: People don't interact when they are
sleeping.

We thus conclude that are differences in how, when, for how long,
and whom people meet and interact with in our data. Furthermore, the
activities and interaction of the students are not random but follow
identifiable patterns. Students socialize more at home and at places
that are important to them, they do not visit the university during
the weekend, and are more likely to be out during the weekend than
at home. Taken at it's face value those findings are rather unexciting,
but we will use those differences for predicting who will meet whom
in the next section. 


\section{\label{sec:Prediction}Link prediction}

Random Forests have been consistently shown to perform well in link-prediction
tasks (for example X and Y) and we thus opt to use them for our prediction
task as well. We are particularly interested in the drivers of social
interaction or in other words what set of features gives us the best
prediction and to a lesser extent in evaluating different classifiers
for the prediction task at hand.


\subsection{Choosing DT and Dt}

As Yang et al () have shown setting the length of DT and Dt has an
impact on the performance of the resulting link prediction. Yang et
al () have proposed to use the time series of the density of the network
as a guide for selecting DT and we mostly follow their approach here.
When looking at the density time series (figure density time series)
one can clearly identify a weekly pattern, but also seasonal effects.
Periods of low density either coincide with holidays or with exam
periods and thus ideally we chose a DT of 14 days to mitigate the
negative effects of seasonality. 

Contrary to Yang et al () we did not opt to use the average duration
of an interaction as Dt. While the average length of interaction is
around 806 seconds, we chose to instead use several hours as Dt. As
this allows us to predict whether there will be an edge between u,v
e V, but also to distinguish between the strength of the tie over
the course of a longer time intervall. This is important as we are
much more interested in the drivers of interaction than in correctly
predicting chance encounters. A shorter Dt would not allow us to predict
the length and thus the nature of the interaction. Furthermore, compare
the entropy of different length of encounters in table X. We can see
that longer encounters are have a consequentially lower entropy value
than shorter ones, the ones we are most interested in. Encounters
of an undefined length make up around 0.49 of all encounters, i.e.
encounters that appear in only sampling interval of the smartphone
and we thus cannot assign a meaningful duration.


\subsection{Search space}

Usually researchers have restricted their search space for new ties
as there are almost N potential candidates in sparse social networks.
The complexity of any algorithm that searches in an unrestricted space
is thus O(N\textasciicircum{}2). Common ways to deal with this are
either considering only the set of friends of friends, ``place-friends''
(), ``mobility'' friends () as potential candidates for new ties.
However, our network is small enough that is still computationally
feasible to consider all possible pairs of nodes. Furthermore, we
can observe a lot of change in the structure of the graph between
time points (figure X) and restricting the search space would exclude
several potential candidates at each time step. 


\subsection{Feature vectors}


\subsubsection{Baseline features}

As our baseline features for all subsequent models we include \emph{recency},
the amount of elapsed time since the last meeting, \emph{activeness,
}how often two nodes interacted (Quotes from the Yang paper) and how
much time spent two nodes spent together during the training period.


\subsubsection{Context features}

We also include several features pertaining to the setting wherein
two nodes meet. These can be split into features relating to time,
space, and the social realm. The time related features pertain to
capture weekly behavioral patterns. Let M be the set of all meetings
between two nodes in the training period. We then include a vector(hour-of-day(M)),
vector((hour-of-week(M)), vector((day-of-week(M)).

We also include min(\emph{place entropy)} (quote) {[}should this be
a vector?{]} of the meetings as we reason that there is a difference
in whether two people meet at a place a lot of people visit and thus
with high place entropy or at quieter place with low place entropy.
Or in other words, if two student meet at the university then this
probably does not tell us that much as a lot of people are meeting
there, but if two people meet at their respective homes then this
is a much more unlikely event.

We also include the \emph{max(relative importance)} of a place measured
as the amount of time a student spends there {[}define. Train wreck
sentence{]}. 

{[}And we do something here and I have to say something about this...
time spent at context\_i with peer{]}

But it is not only the physical qualities of the place that might
{[}do something{]} but the also the social setting an interaction
occurs. If two students meet at the university during a course this
does not tell us much, but if two students meet alone on the campus
there is a higher likelihood that they are socializing. We thus include
the number of other people present min, avg, max(number of people).

{[}I have to re-think this one too:{]} spatial triadic closure

candidates spatial triadic closure


\subsubsection{Network features}


\subsection{Null Model}

{[}null model{]} As a benchmark to test our predictions against we
also developed a null model for a time-evolving weighted interaction
graph with dissolving ties. The null model asserts that change between
time-points in G is happening randomly, while it adheres to the true
amount of change of the graph between time-points. ``True change''
in our case means created ties ($E_{t+\triangle t}\backslash E_{t}$)
as well as dissolved ties ($E_{t+\triangle t}\backslash E_{t}$) for
each class. Where we take the probability that a tie changes classes
between time-points - $P(x_{t+\triangle t}|y_{t})$ where x, y E \{0,1,2,3\}
- from the observation of the actual change between $G_{t}$ and $G_{t+\triangle t}$.

We use the first one to refine our hypothesis and to develop our model,
while we use the second dataset to test our model. In this way we
avoid {[}self-biasing{]} our analysis.


\subsection{Experimental Setup}

We developed and tested it our model on our smaller dataset before
running the ``completed'' model on our second, larger and independent
dataset. This way we avoid biasing ourselves and developing our model
to fit our data. We in this way ensure that 


\section{\label{sec:Findings}Findings}


\subsection{{[}Findings of the general model{]}}


\subsection{{[}Multi-class vs single-class case{]}}


\subsection{{[}Findings for the different models{]}}


\subsection{{[}Performance of the node-only model{]}}


\subsection{{[}Performance of several different T lengths{]}}


\subsection{{[}network prediction performance{]}}


\section{\label{sec:Conclusion}Conclusion/Discussion}
\end{document}
